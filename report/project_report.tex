\documentclass{article}
\usepackage{parskip}
\usepackage{indentfirst}
\setlength{\parindent}{1cm}
\renewcommand{\baselinestretch}{1.4}

\usepackage{graphicx}

\usepackage{hyperref}
\hypersetup{
    colorlinks=true,
    linkcolor=blue,
    filecolor=blue,      
    urlcolor=blue,
}
 
\usepackage[
    top    = 3cm,
    bottom = 2cm,
    left   = 2.8cm,
    right  = 2.8cm]{geometry}
\usepackage[utf8]{inputenc}

\usepackage{fancyhdr}
\pagestyle{fancy}
\renewcommand{\headrulewidth}{0pt}
\fancyhf{}
\lhead{PHYS3051: Lab Report 1}
\rhead{Desiree Vogt-Lee, 44354471}
\cfoot{\thepage}

\title{Implementing the Klein Gordon Equation on a Quantum Computer}
\author{Desiree Vogt-Lee}
\date{June 2019}

\begin{document}
	\maketitle
	\begin{center}
	    All components of this project including code, Jupyter Notebooks, LaTeX files and images can be found in my \href{https://github.com/desireevl/PHYS3051}{Github repository}, please check this out for reference.
	\end{center}


\section{The Klein Gordon Equation}

motivation 

\section{Background}
\subsection{Quantum Computing}
\subsection{Gates}

\href{https://desireevl.github.io/archive/2019/04/03/quantum-intro.html}{blog}


\section{The Circuits}
The barriers that are seem in the image of the circuit serve the purpose of
\subsection{Quantum Fourier Transform and its Inverse}


\section{Results}
\subsection{Kapil, M. et al. (2018) Results}

\subsection{My Results}

\section{References}
[1] Wolfram Alpha (2019). \textit{Wolfram Alpha Search}. Retrieved from
\begin{verbatim}
http
\end{verbatim}




\end{document}
